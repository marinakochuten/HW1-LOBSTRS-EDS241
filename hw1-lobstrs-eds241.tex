% Options for packages loaded elsewhere
\PassOptionsToPackage{unicode}{hyperref}
\PassOptionsToPackage{hyphens}{url}
%
\documentclass[
]{article}
\usepackage{amsmath,amssymb}
\usepackage{iftex}
\ifPDFTeX
  \usepackage[T1]{fontenc}
  \usepackage[utf8]{inputenc}
  \usepackage{textcomp} % provide euro and other symbols
\else % if luatex or xetex
  \usepackage{unicode-math} % this also loads fontspec
  \defaultfontfeatures{Scale=MatchLowercase}
  \defaultfontfeatures[\rmfamily]{Ligatures=TeX,Scale=1}
\fi
\usepackage{lmodern}
\ifPDFTeX\else
  % xetex/luatex font selection
\fi
% Use upquote if available, for straight quotes in verbatim environments
\IfFileExists{upquote.sty}{\usepackage{upquote}}{}
\IfFileExists{microtype.sty}{% use microtype if available
  \usepackage[]{microtype}
  \UseMicrotypeSet[protrusion]{basicmath} % disable protrusion for tt fonts
}{}
\makeatletter
\@ifundefined{KOMAClassName}{% if non-KOMA class
  \IfFileExists{parskip.sty}{%
    \usepackage{parskip}
  }{% else
    \setlength{\parindent}{0pt}
    \setlength{\parskip}{6pt plus 2pt minus 1pt}}
}{% if KOMA class
  \KOMAoptions{parskip=half}}
\makeatother
\usepackage{xcolor}
\usepackage[margin=1in]{geometry}
\usepackage{color}
\usepackage{fancyvrb}
\newcommand{\VerbBar}{|}
\newcommand{\VERB}{\Verb[commandchars=\\\{\}]}
\DefineVerbatimEnvironment{Highlighting}{Verbatim}{commandchars=\\\{\}}
% Add ',fontsize=\small' for more characters per line
\usepackage{framed}
\definecolor{shadecolor}{RGB}{248,248,248}
\newenvironment{Shaded}{\begin{snugshade}}{\end{snugshade}}
\newcommand{\AlertTok}[1]{\textcolor[rgb]{0.94,0.16,0.16}{#1}}
\newcommand{\AnnotationTok}[1]{\textcolor[rgb]{0.56,0.35,0.01}{\textbf{\textit{#1}}}}
\newcommand{\AttributeTok}[1]{\textcolor[rgb]{0.13,0.29,0.53}{#1}}
\newcommand{\BaseNTok}[1]{\textcolor[rgb]{0.00,0.00,0.81}{#1}}
\newcommand{\BuiltInTok}[1]{#1}
\newcommand{\CharTok}[1]{\textcolor[rgb]{0.31,0.60,0.02}{#1}}
\newcommand{\CommentTok}[1]{\textcolor[rgb]{0.56,0.35,0.01}{\textit{#1}}}
\newcommand{\CommentVarTok}[1]{\textcolor[rgb]{0.56,0.35,0.01}{\textbf{\textit{#1}}}}
\newcommand{\ConstantTok}[1]{\textcolor[rgb]{0.56,0.35,0.01}{#1}}
\newcommand{\ControlFlowTok}[1]{\textcolor[rgb]{0.13,0.29,0.53}{\textbf{#1}}}
\newcommand{\DataTypeTok}[1]{\textcolor[rgb]{0.13,0.29,0.53}{#1}}
\newcommand{\DecValTok}[1]{\textcolor[rgb]{0.00,0.00,0.81}{#1}}
\newcommand{\DocumentationTok}[1]{\textcolor[rgb]{0.56,0.35,0.01}{\textbf{\textit{#1}}}}
\newcommand{\ErrorTok}[1]{\textcolor[rgb]{0.64,0.00,0.00}{\textbf{#1}}}
\newcommand{\ExtensionTok}[1]{#1}
\newcommand{\FloatTok}[1]{\textcolor[rgb]{0.00,0.00,0.81}{#1}}
\newcommand{\FunctionTok}[1]{\textcolor[rgb]{0.13,0.29,0.53}{\textbf{#1}}}
\newcommand{\ImportTok}[1]{#1}
\newcommand{\InformationTok}[1]{\textcolor[rgb]{0.56,0.35,0.01}{\textbf{\textit{#1}}}}
\newcommand{\KeywordTok}[1]{\textcolor[rgb]{0.13,0.29,0.53}{\textbf{#1}}}
\newcommand{\NormalTok}[1]{#1}
\newcommand{\OperatorTok}[1]{\textcolor[rgb]{0.81,0.36,0.00}{\textbf{#1}}}
\newcommand{\OtherTok}[1]{\textcolor[rgb]{0.56,0.35,0.01}{#1}}
\newcommand{\PreprocessorTok}[1]{\textcolor[rgb]{0.56,0.35,0.01}{\textit{#1}}}
\newcommand{\RegionMarkerTok}[1]{#1}
\newcommand{\SpecialCharTok}[1]{\textcolor[rgb]{0.81,0.36,0.00}{\textbf{#1}}}
\newcommand{\SpecialStringTok}[1]{\textcolor[rgb]{0.31,0.60,0.02}{#1}}
\newcommand{\StringTok}[1]{\textcolor[rgb]{0.31,0.60,0.02}{#1}}
\newcommand{\VariableTok}[1]{\textcolor[rgb]{0.00,0.00,0.00}{#1}}
\newcommand{\VerbatimStringTok}[1]{\textcolor[rgb]{0.31,0.60,0.02}{#1}}
\newcommand{\WarningTok}[1]{\textcolor[rgb]{0.56,0.35,0.01}{\textbf{\textit{#1}}}}
\usepackage{graphicx}
\makeatletter
\def\maxwidth{\ifdim\Gin@nat@width>\linewidth\linewidth\else\Gin@nat@width\fi}
\def\maxheight{\ifdim\Gin@nat@height>\textheight\textheight\else\Gin@nat@height\fi}
\makeatother
% Scale images if necessary, so that they will not overflow the page
% margins by default, and it is still possible to overwrite the defaults
% using explicit options in \includegraphics[width, height, ...]{}
\setkeys{Gin}{width=\maxwidth,height=\maxheight,keepaspectratio}
% Set default figure placement to htbp
\makeatletter
\def\fps@figure{htbp}
\makeatother
\setlength{\emergencystretch}{3em} % prevent overfull lines
\providecommand{\tightlist}{%
  \setlength{\itemsep}{0pt}\setlength{\parskip}{0pt}}
\setcounter{secnumdepth}{-\maxdimen} % remove section numbering
\usepackage{booktabs}
\usepackage{caption}
\usepackage{longtable}
\usepackage{colortbl}
\usepackage{array}
\usepackage{anyfontsize}
\usepackage{multirow}
\usepackage{wrapfig}
\usepackage{float}
\usepackage{pdflscape}
\usepackage{tabu}
\usepackage{threeparttable}
\usepackage{threeparttablex}
\usepackage[normalem]{ulem}
\usepackage{makecell}
\usepackage{xcolor}
\usepackage{graphicx}
\usepackage{siunitx}
\usepackage{hhline}
\usepackage{calc}
\usepackage{tabularx}
\usepackage{adjustbox}
\usepackage{hyperref}
\ifLuaTeX
  \usepackage{selnolig}  % disable illegal ligatures
\fi
\usepackage{bookmark}
\IfFileExists{xurl.sty}{\usepackage{xurl}}{} % add URL line breaks if available
\urlstyle{same}
\hypersetup{
  pdftitle={Assignment 1: California Spiny Lobster Abundance (Panulirus Interruptus)},
  pdfauthor={EDS 241},
  hidelinks,
  pdfcreator={LaTeX via pandoc}}

\title{Assignment 1: California Spiny Lobster Abundance (\emph{Panulirus
Interruptus})}
\usepackage{etoolbox}
\makeatletter
\providecommand{\subtitle}[1]{% add subtitle to \maketitle
  \apptocmd{\@title}{\par {\large #1 \par}}{}{}
}
\makeatother
\subtitle{Assessing the Impact of Marine Protected Areas (MPAs) at 5
Reef Sites in Santa Barbara County}
\author{EDS 241}
\date{1/8/2024 (Due 1/22)}

\begin{document}
\maketitle

\begin{center}\rule{0.5\linewidth}{0.5pt}\end{center}

\includegraphics{figures/spiny2.jpg}

\begin{center}\rule{0.5\linewidth}{0.5pt}\end{center}

\subsubsection{Assignment instructions:}\label{assignment-instructions}

\begin{itemize}
\item
  Working with partners to troubleshoot code and concepts is encouraged!
  If you work with a partner, please list their name next to yours at
  the top of your assignment so Annie and I can easily see who
  collaborated.
\item
  All written responses must be written independently (\textbf{in your
  own words}).
\item
  Please follow the question prompts carefully and include only the
  information each question asks in your submitted responses.
\item
  Submit both your knitted document and the associated
  \texttt{RMarkdown} or \texttt{Quarto} file.
\item
  Your knitted presentation should meet the quality you'd submit to
  research colleagues or feel confident sharing publicly. Refer to the
  rubric for details about presentation standards.
\end{itemize}

\textbf{Assignment submission (Marina Kochuten):}
\_\_\_\_\_\_\_\_\_\_\_\_\_\_\_\_\_\_\_\_\_\_\_\_\_\_\_\_\_\_\_\_\_\_\_\_\_\_

\begin{center}\rule{0.5\linewidth}{0.5pt}\end{center}

\begin{Shaded}
\begin{Highlighting}[]
\CommentTok{\# Load libraries}
\FunctionTok{library}\NormalTok{(tidyverse)}
\FunctionTok{library}\NormalTok{(here)}
\FunctionTok{library}\NormalTok{(janitor)}
\FunctionTok{library}\NormalTok{(estimatr)  }
\FunctionTok{library}\NormalTok{(performance)}
\FunctionTok{library}\NormalTok{(jtools)}
\FunctionTok{library}\NormalTok{(gt)}
\FunctionTok{library}\NormalTok{(gtsummary)}
\FunctionTok{library}\NormalTok{(MASS) }\DocumentationTok{\#\# }\AlertTok{NOTE}\DocumentationTok{: The \textasciigrave{}select()\textasciigrave{} function is masked. Use: \textasciigrave{}dplyr::select()\textasciigrave{} \#\#}
\FunctionTok{library}\NormalTok{(interactions) }
\end{Highlighting}
\end{Shaded}

\begin{center}\rule{0.5\linewidth}{0.5pt}\end{center}

\paragraph{DATA SOURCE:}\label{data-source}

Reed D. 2019. SBC LTER: Reef: Abundance, size and fishing effort for
California Spiny Lobster (Panulirus interruptus), ongoing since 2012.
Environmental Data Initiative.
\url{https://doi.org/10.6073/pasta/a593a675d644fdefb736750b291579a0}.
Dataset accessed 11/17/2019.

\begin{center}\rule{0.5\linewidth}{0.5pt}\end{center}

\subsubsection{\texorpdfstring{\textbf{Introduction}}{Introduction}}\label{introduction}

You're about to dive into some deep data collected from five reef sites
in Santa Barbara County, all about the abundance of California spiny
lobsters! 🦞 Data was gathered by divers annually from 2012 to 2018
across Naples, Mohawk, Isla Vista, Carpinteria, and Arroyo Quemado
reefs.

Why lobsters? Well, this sample provides an opportunity to evaluate the
impact of Marine Protected Areas (MPAs) established on January 1, 2012
(Reed, 2019). Of these five reefs, Naples, and Isla Vista are MPAs,
while the other three are not protected (non-MPAs). Comparing lobster
health between these protected and non-protected areas gives us the
chance to study how commercial and recreational fishing might impact
these ecosystems.

We will consider the MPA sites the \texttt{treatment} group and use
regression methods to explore whether protecting these reefs really
makes a difference compared to non-MPA sites (our control group). In
this assignment, we'll think deeply about which causal inference
assumptions hold up under the research design and identify where they
fall short.

Let's break it down step by step and see what the data reveals! 📊

\includegraphics{figures/map-5reefs.png}

\begin{center}\rule{0.5\linewidth}{0.5pt}\end{center}

Step 1: Anticipating potential sources of selection bias

\textbf{a.} Do the control sites (Arroyo Quemado, Carpenteria, and
Mohawk) provide a strong counterfactual for our treatment sites (Naples,
Isla Vista)? Write a paragraph making a case for why this comparison is
centris paribus or whether selection bias is likely (be specific!).

\begin{itemize}
\tightlist
\item
  \textbf{I would say that the control sites do provide a strong
  counterfactual for our treatment sites. I would expect that lobster
  populations between the MPA sites and non-MPA sites were likely equal
  before the establishment of MPAs. The sites chosen for this experiment
  all fall well within the range of the California Spiny Lobster and
  they are all at similar locations geographically.}
\end{itemize}

\begin{center}\rule{0.5\linewidth}{0.5pt}\end{center}

Step 2: Read \& wrangle data

\textbf{a.} Read in the raw data. Name the data.frame (\texttt{df})
\texttt{rawdata}

\textbf{b.} Use the function \texttt{clean\_names()} from the
\texttt{janitor} package

\begin{Shaded}
\begin{Highlighting}[]
\CommentTok{\# HINT: check for coding of missing values (\textasciigrave{}na = "{-}99999"\textasciigrave{})}

\CommentTok{\# read in the data, convert "{-}99999" to NA, and clean col names}
\NormalTok{rawdata }\OtherTok{\textless{}{-}} \FunctionTok{read\_csv}\NormalTok{(}\FunctionTok{here}\NormalTok{(}\StringTok{"data"}\NormalTok{, }\StringTok{"spiny\_abundance\_sb\_18.csv"}\NormalTok{), }\AttributeTok{na =} \StringTok{"{-}99999"}\NormalTok{) }\SpecialCharTok{|\textgreater{}}
    \FunctionTok{clean\_names}\NormalTok{()}
\end{Highlighting}
\end{Shaded}

\textbf{c.} Create a new \texttt{df} named \texttt{tidyata}. Using the
variable \texttt{site} (reef location) create a new variable
\texttt{reef} as a \texttt{factor} and add the following labels in the
order listed (i.e., re-order the \texttt{levels}):

\begin{verbatim}
"Arroyo Quemado", "Carpenteria", "Mohawk", "Isla Vista",  "Naples"
\end{verbatim}

\begin{Shaded}
\begin{Highlighting}[]
\CommentTok{\# Add long lables to our sites and save in a col named reef}
\NormalTok{tidydata }\OtherTok{\textless{}{-}}\NormalTok{ rawdata }\SpecialCharTok{|\textgreater{}}
    \FunctionTok{mutate}\NormalTok{(}\AttributeTok{reef =} \FunctionTok{factor}\NormalTok{(site, }
                         \AttributeTok{levels =} \FunctionTok{c}\NormalTok{(}\StringTok{"AQUE"}\NormalTok{, }\StringTok{"CARP"}\NormalTok{, }\StringTok{"MOHK"}\NormalTok{, }\StringTok{"IVEE"}\NormalTok{, }\StringTok{"NAPL"}\NormalTok{), }
                         \AttributeTok{labels =} \FunctionTok{c}\NormalTok{(}\StringTok{"Arroyo Quemado"}\NormalTok{, }\StringTok{"Carpenteria"}\NormalTok{, }\StringTok{"Mohawk"}\NormalTok{, }
                                    \StringTok{"Isla Vista"}\NormalTok{,  }\StringTok{"Naples"}\NormalTok{)))}
\end{Highlighting}
\end{Shaded}

Create new \texttt{df} named \texttt{spiny\_counts}

\textbf{d.} Create a new variable \texttt{counts} to allow for an
analysis of lobster counts where the unit-level of observation is the
total number of observed lobsters per \texttt{site}, \texttt{year} and
\texttt{transect}.

\begin{itemize}
\tightlist
\item
  Create a variable \texttt{mean\_size} from the variable
  \texttt{size\_mm}
\item
  NOTE: The variable \texttt{counts} should have values which are
  integers (whole numbers).
\item
  Make sure to account for missing cases (\texttt{na})!
\end{itemize}

\textbf{e.} Create a new variable \texttt{mpa} with levels \texttt{MPA}
and \texttt{non\_MPA}. For our regression analysis create a numerical
variable \texttt{treat} where MPA sites are coded \texttt{1} and
non\_MPA sites are coded \texttt{0}

\begin{Shaded}
\begin{Highlighting}[]
\CommentTok{\#HINT(d): Use \textasciigrave{}group\_by()\textasciigrave{} \& \textasciigrave{}summarize()\textasciigrave{} to provide the total number of lobsters observed at each site{-}year{-}transect row{-}observation. }

\CommentTok{\#HINT(e): Use \textasciigrave{}case\_when()\textasciigrave{} to create the 3 new variable columns}

\CommentTok{\# assign each site either mpa or non{-}mpa, and 1 or 0}
\NormalTok{spiny\_counts }\OtherTok{\textless{}{-}}\NormalTok{ tidydata }\SpecialCharTok{|\textgreater{}}
    \FunctionTok{group\_by}\NormalTok{(site, year, transect) }\SpecialCharTok{|\textgreater{}}
    \FunctionTok{summarise}\NormalTok{(}\AttributeTok{count =} \FunctionTok{sum}\NormalTok{(count, }\AttributeTok{na.rm =} \ConstantTok{TRUE}\NormalTok{), }\AttributeTok{mean\_size =} \FunctionTok{mean}\NormalTok{(size\_mm, }\AttributeTok{na.rm =} \ConstantTok{TRUE}\NormalTok{)) }\SpecialCharTok{|\textgreater{}}
    \FunctionTok{mutate}\NormalTok{(}\AttributeTok{mpa =} \FunctionTok{case\_when}\NormalTok{(site }\SpecialCharTok{\%in\%} \FunctionTok{c}\NormalTok{(}\StringTok{"IVEE"}\NormalTok{, }\StringTok{"NAPL"}\NormalTok{) }\SpecialCharTok{\textasciitilde{}} \StringTok{"MPA"}\NormalTok{,}
                           \AttributeTok{.default =} \StringTok{"non\_MPA"}\NormalTok{)) }\SpecialCharTok{|\textgreater{}}
    \FunctionTok{mutate}\NormalTok{(}\AttributeTok{treat =} \FunctionTok{case\_when}\NormalTok{(mpa }\SpecialCharTok{==} \StringTok{"MPA"} \SpecialCharTok{\textasciitilde{}} \DecValTok{1}\NormalTok{,}
                             \AttributeTok{.default =} \DecValTok{0}\NormalTok{)) }\SpecialCharTok{|\textgreater{}}
    \FunctionTok{ungroup}\NormalTok{()}
\end{Highlighting}
\end{Shaded}

\begin{quote}
NOTE: This step is crucial to the analysis. Check with a friend or come
to TA/instructor office hours to make sure the counts are coded
correctly!
\end{quote}

\begin{center}\rule{0.5\linewidth}{0.5pt}\end{center}

Step 3: Explore \& visualize data

\textbf{a.} Take a look at the data! Get familiar with the data in each
\texttt{df} format (\texttt{tidydata}, \texttt{spiny\_counts})

\textbf{b.} We will focus on the variables \texttt{count},
\texttt{year}, \texttt{site}, and \texttt{treat}(\texttt{mpa}) to model
lobster abundance. Create the following 4 plots using a different method
each time from the 6 options provided. Add a layer (\texttt{geom}) to
each of the plots including informative descriptive statistics (you
choose; e.g., mean, median, SD, quartiles, range). Make sure each plot
dimension is clearly labeled (e.g., axes, groups).

\begin{itemize}
\tightlist
\item
  \href{https://r-charts.com/distribution/density-plot-group-ggplot2}{Density
  plot}
\item
  \href{https://r-charts.com/distribution/ggridges/}{Ridge plot}
\item
  \href{https://ggplot2.tidyverse.org/reference/geom_jitter.html}{Jitter
  plot}
\item
  \href{https://r-charts.com/distribution/violin-plot-group-ggplot2}{Violin
  plot}
\item
  \href{https://r-charts.com/distribution/histogram-density-ggplot2/}{Histogram}
\item
  \href{https://r-charts.com/distribution/beeswarm/}{Beeswarm}
\end{itemize}

Create plots displaying the distribution of lobster \textbf{counts}:

\begin{enumerate}
\def\labelenumi{\arabic{enumi})}
\tightlist
\item
  grouped by reef site\\
\item
  grouped by MPA status
\item
  grouped by year
\end{enumerate}

Create a plot of lobster \textbf{size} :

\begin{enumerate}
\def\labelenumi{\arabic{enumi})}
\setcounter{enumi}{3}
\tightlist
\item
  You choose the grouping variable(s)!
\end{enumerate}

\begin{Shaded}
\begin{Highlighting}[]
\CommentTok{\# plot 1: Ridge plot grouped by reef site}
\FunctionTok{library}\NormalTok{(ggridges)}

\NormalTok{spiny\_counts }\SpecialCharTok{\%\textgreater{}\%} 
    \FunctionTok{ggplot}\NormalTok{(}\FunctionTok{aes}\NormalTok{(}\AttributeTok{x =}\NormalTok{ count, }\AttributeTok{y =}\NormalTok{ site)) }\SpecialCharTok{+}
    \FunctionTok{geom\_density\_ridges2}\NormalTok{(}\AttributeTok{quantile\_lines =} \ConstantTok{TRUE}\NormalTok{,}
                         \AttributeTok{quantiles =} \DecValTok{2}\NormalTok{,}
                         \AttributeTok{alpha =} \FloatTok{0.3}\NormalTok{,}
                         \AttributeTok{fill =} \StringTok{"coral"}\NormalTok{) }\SpecialCharTok{+}
    \FunctionTok{xlab}\NormalTok{(}\StringTok{"California Spiny Lobster Count"}\NormalTok{) }\SpecialCharTok{+}
    \FunctionTok{ylab}\NormalTok{(}\StringTok{"Reef Site"}\NormalTok{) }\SpecialCharTok{+}
    \FunctionTok{labs}\NormalTok{(}\AttributeTok{title =} \StringTok{"Distribution of California Spiny Lobster Counts by Reef Site \& Median Counts"}\NormalTok{) }\SpecialCharTok{+}
    \FunctionTok{theme\_minimal}\NormalTok{()}
\end{Highlighting}
\end{Shaded}

\includegraphics{hw1-lobstrs-eds241_files/figure-latex/unnamed-chunk-5-1.pdf}

\begin{Shaded}
\begin{Highlighting}[]
\CommentTok{\# Plot 2: Jitter plot of counts grouped by MPA status}
\NormalTok{spiny\_counts }\SpecialCharTok{|\textgreater{}}
    \FunctionTok{ggplot}\NormalTok{(}\FunctionTok{aes}\NormalTok{(}\AttributeTok{y =}\NormalTok{ count, }\AttributeTok{x =}\NormalTok{ mpa)) }\SpecialCharTok{+}
    \FunctionTok{geom\_boxplot}\NormalTok{(}\AttributeTok{width =} \FloatTok{0.2}\NormalTok{, }\AttributeTok{outliers =} \ConstantTok{FALSE}\NormalTok{) }\SpecialCharTok{+}
    \FunctionTok{geom\_jitter}\NormalTok{(}\AttributeTok{width =} \FloatTok{0.3}\NormalTok{, }\AttributeTok{color =} \StringTok{"coral"}\NormalTok{, }\AttributeTok{alpha =} \FloatTok{0.5}\NormalTok{) }\SpecialCharTok{+}
    \FunctionTok{xlab}\NormalTok{(}\StringTok{"MPA Status"}\NormalTok{) }\SpecialCharTok{+}
    \FunctionTok{ylab}\NormalTok{(}\StringTok{"California Spiny Lobster Count"}\NormalTok{) }\SpecialCharTok{+}
    \FunctionTok{labs}\NormalTok{(}\AttributeTok{title =} \StringTok{"Distribution of California Spiny Lobster Counts by MPA status"}\NormalTok{) }\SpecialCharTok{+}
    \FunctionTok{theme\_minimal}\NormalTok{()}
\end{Highlighting}
\end{Shaded}

\includegraphics{hw1-lobstrs-eds241_files/figure-latex/unnamed-chunk-6-1.pdf}

\begin{Shaded}
\begin{Highlighting}[]
\CommentTok{\# Plot 3: Grouped by year}
\NormalTok{spiny\_counts }\SpecialCharTok{\%\textgreater{}\%} 
    \FunctionTok{ggplot}\NormalTok{(}\FunctionTok{aes}\NormalTok{(}\AttributeTok{x =}\NormalTok{ count, }\AttributeTok{y =} \FunctionTok{factor}\NormalTok{(year))) }\SpecialCharTok{+}
    \FunctionTok{geom\_violin}\NormalTok{(}\AttributeTok{trim =} \ConstantTok{TRUE}\NormalTok{) }\SpecialCharTok{+}
    \FunctionTok{stat\_summary}\NormalTok{(}\AttributeTok{fun.y =}\NormalTok{ mean, }\AttributeTok{geom =} \StringTok{"point"}\NormalTok{, }\AttributeTok{color =} \StringTok{"coral"}\NormalTok{, }\AttributeTok{size =} \DecValTok{3}\NormalTok{, }\AttributeTok{alpha =} \FloatTok{0.7}\NormalTok{)  }\SpecialCharTok{+}
    \FunctionTok{xlab}\NormalTok{(}\StringTok{"California Spiny Lobster Count"}\NormalTok{) }\SpecialCharTok{+}
    \FunctionTok{ylab}\NormalTok{(}\StringTok{"Year"}\NormalTok{) }\SpecialCharTok{+}
    \FunctionTok{labs}\NormalTok{(}\AttributeTok{title =} \StringTok{"Distribution of California Spiny Lobster Counts by Year \& Yearly Average Counts"}\NormalTok{) }\SpecialCharTok{+}
    \FunctionTok{theme\_minimal}\NormalTok{()}
\end{Highlighting}
\end{Shaded}

\includegraphics{hw1-lobstrs-eds241_files/figure-latex/unnamed-chunk-7-1.pdf}

\begin{Shaded}
\begin{Highlighting}[]
\CommentTok{\# Plot 4: Plot of lobster size}
\FunctionTok{library}\NormalTok{(ggbeeswarm)}

\NormalTok{spiny\_counts }\SpecialCharTok{|\textgreater{}}
    \FunctionTok{ggplot}\NormalTok{(}\FunctionTok{aes}\NormalTok{(}\AttributeTok{x =}\NormalTok{ mean\_size, }\AttributeTok{fill =}\NormalTok{ mpa)) }\SpecialCharTok{+}
    \FunctionTok{geom\_density}\NormalTok{(}\AttributeTok{alpha =} \FloatTok{0.5}\NormalTok{) }\SpecialCharTok{+}
    \FunctionTok{scale\_fill\_manual}\NormalTok{(}\AttributeTok{values =} \FunctionTok{c}\NormalTok{(}\StringTok{"coral"}\NormalTok{, }\StringTok{"lightyellow"}\NormalTok{)) }\SpecialCharTok{+}
    \FunctionTok{xlab}\NormalTok{(}\StringTok{"California Spiny Lobster Mean Size (mm)"}\NormalTok{) }\SpecialCharTok{+}
    \FunctionTok{ylab}\NormalTok{(}\StringTok{"Density"}\NormalTok{) }\SpecialCharTok{+}
    \FunctionTok{labs}\NormalTok{(}\AttributeTok{title =} \StringTok{"Distribution of CA Spiny Lobster Size by MPA Status"}\NormalTok{) }\SpecialCharTok{+}
    \FunctionTok{theme\_minimal}\NormalTok{() }\SpecialCharTok{+}
    \FunctionTok{geom\_vline}\NormalTok{(}\AttributeTok{xintercept =} \FunctionTok{mean}\NormalTok{(spiny\_counts}\SpecialCharTok{$}\NormalTok{mean\_size, }\AttributeTok{na.rm =} \ConstantTok{TRUE}\NormalTok{), }\AttributeTok{color =} \StringTok{"black"}\NormalTok{) }\SpecialCharTok{+}
    \FunctionTok{geom\_label}\NormalTok{(}\FunctionTok{aes}\NormalTok{(}\AttributeTok{x =} \FloatTok{61.5}\NormalTok{, }
                  \AttributeTok{y =} \FloatTok{0.065}\NormalTok{,}
                  \AttributeTok{label =} \FunctionTok{paste}\NormalTok{(}\StringTok{"Overall Lobster Mean Size = "}\NormalTok{, }\FunctionTok{round}\NormalTok{(}\FunctionTok{mean}\NormalTok{(spiny\_counts}\SpecialCharTok{$}\NormalTok{mean\_size, }\AttributeTok{na.rm =} \ConstantTok{TRUE}\NormalTok{), }\DecValTok{2}\NormalTok{), }\StringTok{"mm"}\NormalTok{)),}
               \AttributeTok{size =} \DecValTok{3}\NormalTok{,}
               \AttributeTok{show.legend =} \ConstantTok{FALSE}\NormalTok{)}
\end{Highlighting}
\end{Shaded}

\includegraphics{hw1-lobstrs-eds241_files/figure-latex/unnamed-chunk-8-1.pdf}

\textbf{c.} Compare means of the outcome by treatment group. Using the
\texttt{tbl\_summary()} function from the package
\href{https://www.danieldsjoberg.com/gtsummary/articles/tbl_summary.html}{\texttt{gt\_summary}}

\begin{Shaded}
\begin{Highlighting}[]
\CommentTok{\# USE: gt\_summary::tbl\_summary()}
\NormalTok{ spiny\_counts }\SpecialCharTok{|\textgreater{}} 
\NormalTok{    dplyr}\SpecialCharTok{::}\FunctionTok{select}\NormalTok{(count, mean\_size, mpa) }\SpecialCharTok{|\textgreater{}}
    \FunctionTok{tbl\_summary}\NormalTok{(}\AttributeTok{by =}\NormalTok{ mpa,}
                \AttributeTok{statistic =} \FunctionTok{list}\NormalTok{(}\FunctionTok{all\_continuous}\NormalTok{() }\SpecialCharTok{\textasciitilde{}} \StringTok{"\{mean\}"}\NormalTok{)) }\SpecialCharTok{|\textgreater{}}
    \FunctionTok{modify\_caption}\NormalTok{(}\StringTok{"**Comparing CA Spiny Lobster mean counts and sizes between MPA and non{-}MPA sites**"}\NormalTok{)}
\end{Highlighting}
\end{Shaded}

\begin{table}[!t]
\caption{\label{tab:unnamed-chunk-9}\textbf{Comparing CA Spiny Lobster mean counts and sizes between MPA and non-MPA sites}} 
\fontsize{12.0pt}{14.4pt}\selectfont
\begin{tabular*}{\linewidth}{@{\extracolsep{\fill}}lcc}
\toprule
\textbf{Characteristic} & \textbf{MPA}  N = 119\textsuperscript{\textit{1}} & \textbf{non\_MPA}  N = 133\textsuperscript{\textit{1}} \\ 
\midrule\addlinespace[2.5pt]
count & 28 & 23 \\ 
mean\_size & 76 & 73 \\ 
    Unknown & 12 & 15 \\ 
\bottomrule
\end{tabular*}
\begin{minipage}{\linewidth}
\textsuperscript{\textit{1}}Mean\\
\end{minipage}
\end{table}

\begin{center}\rule{0.5\linewidth}{0.5pt}\end{center}

Step 4: OLS regression- building intuition

\textbf{a.} Start with a simple OLS estimator of lobster counts
regressed on treatment. Use the function \texttt{summ()} from the
\href{https://jtools.jacob-long.com/}{\texttt{jtools}} package to print
the OLS output

\textbf{b.} Interpret the intercept \& predictor coefficients \emph{in
your own words}. Use full sentences and write your interpretation of the
regression results to be as clear as possible to a non-academic
audience. - \textbf{At the non-MPA reef sites, there are about 23 CA
Spiny Lobsters. At MPA designated reef sites, there are about 5 more CA
Spiny Lobsters for a total count of 28 lobsters.}

\begin{Shaded}
\begin{Highlighting}[]
\CommentTok{\# }\AlertTok{NOTE}\CommentTok{: We will not evaluate/interpret model fit in this assignment (e.g., R{-}square)}

\NormalTok{m1\_ols }\OtherTok{\textless{}{-}} \FunctionTok{lm}\NormalTok{(count }\SpecialCharTok{\textasciitilde{}}\NormalTok{ treat,}
             \AttributeTok{data =}\NormalTok{ spiny\_counts)}

\FunctionTok{summ}\NormalTok{(m1\_ols, }\AttributeTok{model.fit =} \ConstantTok{FALSE}\NormalTok{) }
\end{Highlighting}
\end{Shaded}

\begin{table}[!h]
\centering
\begin{tabular}{lr}
\toprule
\cellcolor{gray!10}{Observations} & \cellcolor{gray!10}{252}\\
Dependent variable & count\\
\cellcolor{gray!10}{Type} & \cellcolor{gray!10}{OLS linear regression}\\
\bottomrule
\end{tabular}
\end{table}  \begin{table}[!h]
\centering
\begin{threeparttable}
\begin{tabular}{lrrrr}
\toprule
  & Est. & S.E. & t val. & p\\
\midrule
\cellcolor{gray!10}{(Intercept)} & \cellcolor{gray!10}{22.73} & \cellcolor{gray!10}{3.57} & \cellcolor{gray!10}{6.36} & \cellcolor{gray!10}{0.00}\\
treat & 5.36 & 5.20 & 1.03 & 0.30\\
\bottomrule
\end{tabular}
\begin{tablenotes}
\item Standard errors: OLS
\end{tablenotes}
\end{threeparttable}
\end{table}

\textbf{c.} Check the model assumptions using the \texttt{check\_model}
function from the \texttt{performance} package

\textbf{d.} Explain the results of the 4 diagnostic plots. Why are we
getting this result?

\begin{Shaded}
\begin{Highlighting}[]
\FunctionTok{check\_model}\NormalTok{(m1\_ols,  }\AttributeTok{check =} \StringTok{"qq"}\NormalTok{ )}
\end{Highlighting}
\end{Shaded}

\includegraphics{hw1-lobstrs-eds241_files/figure-latex/unnamed-chunk-11-1.pdf}

\textbf{The qq plot above shows us that the residuals are not normally
distributed, which violates one of the key assumptions of OLS. This
tells us that OLS may not be the best model for our data.}

\begin{Shaded}
\begin{Highlighting}[]
\FunctionTok{check\_model}\NormalTok{(m1\_ols, }\AttributeTok{check =} \StringTok{"normality"}\NormalTok{)}
\end{Highlighting}
\end{Shaded}

\includegraphics{hw1-lobstrs-eds241_files/figure-latex/unnamed-chunk-12-1.pdf}

\textbf{This plot above showing the distribution of our residuals shows
us, again, that they are not normally distributed, violating a key OLS
assumption. In this case, the residuals are closer to a log-normal
distribution.}

\begin{Shaded}
\begin{Highlighting}[]
\FunctionTok{check\_model}\NormalTok{(m1\_ols, }\AttributeTok{check =} \StringTok{"homogeneity"}\NormalTok{)}
\end{Highlighting}
\end{Shaded}

\includegraphics{hw1-lobstrs-eds241_files/figure-latex/unnamed-chunk-13-1.pdf}

\textbf{Another assumption of OLS is that the residuals have constant
variance. In this plot above, we can see that the variance is not
constant, violating the OLS assumption.}

\begin{Shaded}
\begin{Highlighting}[]
\FunctionTok{check\_model}\NormalTok{(m1\_ols, }\AttributeTok{check =} \StringTok{"pp\_check"}\NormalTok{)}
\end{Highlighting}
\end{Shaded}

\includegraphics{hw1-lobstrs-eds241_files/figure-latex/unnamed-chunk-14-1.pdf}

\textbf{In the figure above, we can see that the actual (observed) data
is not a good fit to what we would expect from the model. Because these
4 plots show us that key assumptions of OLS are not being met, and that
the observed data is not a good match to the model-predicted data, we
can say that OLS is not the best model in this case.}

\begin{center}\rule{0.5\linewidth}{0.5pt}\end{center}

Step 5: Fitting GLMs

\textbf{a.} Estimate a Poisson regression model using the \texttt{glm()}
function

\begin{Shaded}
\begin{Highlighting}[]
\CommentTok{\#HINT1: Incidence Ratio Rate (IRR): Exponentiation of beta returns coefficient which is interpreted as the \textquotesingle{}percent change\textquotesingle{} for a one unit increase in the predictor }

\CommentTok{\#HINT2: For the second glm() argument \textasciigrave{}family\textasciigrave{} use the following specification option \textasciigrave{}family = poisson(link = "log")\textasciigrave{}}

\CommentTok{\# Estimate Poisson regression model}
\NormalTok{m2\_pois }\OtherTok{\textless{}{-}} \FunctionTok{glm}\NormalTok{(count }\SpecialCharTok{\textasciitilde{}}\NormalTok{ treat,}
               \AttributeTok{family =} \FunctionTok{poisson}\NormalTok{(}\AttributeTok{link =} \StringTok{"log"}\NormalTok{),}
               \AttributeTok{data =}\NormalTok{ spiny\_counts)}

\CommentTok{\# Print poisson model output}
\FunctionTok{summ}\NormalTok{(m2\_pois, }\AttributeTok{model.fit =} \ConstantTok{FALSE}\NormalTok{)}
\end{Highlighting}
\end{Shaded}

\begin{table}[!h]
\centering
\begin{tabular}{lr}
\toprule
\cellcolor{gray!10}{Observations} & \cellcolor{gray!10}{252}\\
Dependent variable & count\\
\cellcolor{gray!10}{Type} & \cellcolor{gray!10}{Generalized linear model}\\
Family & poisson\\
\cellcolor{gray!10}{Link} & \cellcolor{gray!10}{log}\\
\bottomrule
\end{tabular}
\end{table}  \begin{table}[!h]
\centering
\begin{threeparttable}
\begin{tabular}{lrrrr}
\toprule
  & Est. & S.E. & z val. & p\\
\midrule
\cellcolor{gray!10}{(Intercept)} & \cellcolor{gray!10}{3.12} & \cellcolor{gray!10}{0.02} & \cellcolor{gray!10}{171.74} & \cellcolor{gray!10}{0.00}\\
treat & 0.21 & 0.03 & 8.44 & 0.00\\
\bottomrule
\end{tabular}
\begin{tablenotes}
\item Standard errors: MLE
\end{tablenotes}
\end{threeparttable}
\end{table}

\begin{Shaded}
\begin{Highlighting}[]
\CommentTok{\# Interpret results as percent change}
\CommentTok{\# Use this way to avoid hard{-}coding? exp(m2\_pois$coefficients[2]) {-} 1}
\FunctionTok{exp}\NormalTok{(}\FloatTok{0.21}\NormalTok{)}\SpecialCharTok{{-}}\DecValTok{1}
\end{Highlighting}
\end{Shaded}

\begin{verbatim}
## [1] 0.2336781
\end{verbatim}

\textbf{b.} Interpret the predictor coefficient in your own words. Use
full sentences and write your interpretation of the results to be as
clear as possible to a non-academic audience.

\begin{itemize}
\tightlist
\item
  \textbf{In this model, there are 23\% more lobsters at MPA reef sites
  than at non-MPA reef sites.}
\end{itemize}

\textbf{c.} Explain the statistical concept of dispersion and
overdispersion in the context of this model.

\begin{itemize}
\tightlist
\item
  \textbf{Poisson regression makes an assumption that variance
  (dispersion) is proportional to the mean. If the data has
  overdispersion, then the variance (dispersion) is greater than the
  mean.}
\end{itemize}

\textbf{d.} Compare results with previous model, explain change in the
significance of the treatment effect

\begin{itemize}
\tightlist
\item
  \textbf{In the previous model (OLS), we saw an increase of about 5
  lobsters in MPA sites compared to non-MPA sites, which in this case
  calculates to about a 23.6\% increase. In the current model (Poisson),
  we see a 23.4\% increase in lobsters in the MPA sites. The new model
  \emph{slightly} decreases the significance of the treatment effect.}
\end{itemize}

\textbf{e.} Check the model assumptions. Explain results.

\begin{itemize}
\tightlist
\item
  \textbf{Checking the model assumptions below, we can see that the
  Poisson model wasn't the best fit. In the pp-check, the prediced
  values do not follow the observed data. In the zero-inflation check,
  the residual variance does not follow the predicted. We can also see
  in the q-q plot that the residuals are not normally distributed.}
\end{itemize}

\begin{Shaded}
\begin{Highlighting}[]
\FunctionTok{check\_model}\NormalTok{(m2\_pois)}
\end{Highlighting}
\end{Shaded}

\includegraphics{hw1-lobstrs-eds241_files/figure-latex/unnamed-chunk-16-1.pdf}

\textbf{f.} Conduct tests for over-dispersion \& zero-inflation. Explain
results.

\begin{Shaded}
\begin{Highlighting}[]
\FunctionTok{check\_overdispersion}\NormalTok{(m2\_pois)}
\end{Highlighting}
\end{Shaded}

\begin{verbatim}
## # Overdispersion test
## 
##        dispersion ratio =    67.033
##   Pearson's Chi-Squared = 16758.289
##                 p-value =   < 0.001
\end{verbatim}

\textbf{This test implies the dispersion (variance) is significantly
larger than the mean!}

\begin{Shaded}
\begin{Highlighting}[]
\FunctionTok{check\_zeroinflation}\NormalTok{(m2\_pois)}
\end{Highlighting}
\end{Shaded}

\begin{verbatim}
## # Check for zero-inflation
## 
##    Observed zeros: 27
##   Predicted zeros: 0
##             Ratio: 0.00
\end{verbatim}

\textbf{This test implies that the model is predicting less zeros than
actually occur in the data (underfitting zeros), which means there is
probable zero-inflation. This model may not be the best fit for our
data.}

\textbf{g.} Fit a negative binomial model using the function glm.nb()
from the package \texttt{MASS} and check model diagnostics

\textbf{h.} In 1-2 sentences explain rationale for fitting this GLM
model.

\textbf{Negative binomial regression is a generalization of Poisson
regression which loosens the restrictive assumption that the variance is
equal to the mean. In the overdispersion test above, we saw that in this
case variance is significantly larger than the mean. Therefore, we
needed to pick a model does not assume equal variance and mean.}

\textbf{i.} Interpret the treatment estimate result in your own words.
Compare with results from the previous model.

\textbf{In this model, there are 23\% more lobsters in our treatment
(MPA) group than our control (non-MPA) group. This effect is the same
that we observed in the previous model.}

\begin{Shaded}
\begin{Highlighting}[]
\CommentTok{\# }\AlertTok{NOTE}\CommentTok{: The \textasciigrave{}glm.nb()\textasciigrave{} function does not require a \textasciigrave{}family\textasciigrave{} argument}

\NormalTok{m3\_nb }\OtherTok{\textless{}{-}} \FunctionTok{glm.nb}\NormalTok{(count }\SpecialCharTok{\textasciitilde{}}\NormalTok{ treat,}
               \AttributeTok{data =}\NormalTok{ spiny\_counts)}

\CommentTok{\# Print negative binomial model output}
\FunctionTok{summ}\NormalTok{(m3\_nb, }\AttributeTok{model.fit =} \ConstantTok{FALSE}\NormalTok{)}
\end{Highlighting}
\end{Shaded}

\begin{table}[!h]
\centering
\begin{tabular}{lr}
\toprule
\cellcolor{gray!10}{Observations} & \cellcolor{gray!10}{252}\\
Dependent variable & count\\
\cellcolor{gray!10}{Type} & \cellcolor{gray!10}{Generalized linear model}\\
Family & Negative Binomial(0.55)\\
\cellcolor{gray!10}{Link} & \cellcolor{gray!10}{log}\\
\bottomrule
\end{tabular}
\end{table}  \begin{table}[!h]
\centering
\begin{threeparttable}
\begin{tabular}{lrrrr}
\toprule
  & Est. & S.E. & z val. & p\\
\midrule
\cellcolor{gray!10}{(Intercept)} & \cellcolor{gray!10}{3.12} & \cellcolor{gray!10}{0.12} & \cellcolor{gray!10}{26.40} & \cellcolor{gray!10}{0.00}\\
treat & 0.21 & 0.17 & 1.23 & 0.22\\
\bottomrule
\end{tabular}
\begin{tablenotes}
\item Standard errors: MLE
\end{tablenotes}
\end{threeparttable}
\end{table}

\begin{Shaded}
\begin{Highlighting}[]
\CommentTok{\# Interpret results as percent change}
\FunctionTok{exp}\NormalTok{(}\FloatTok{0.21}\NormalTok{)}\SpecialCharTok{{-}}\DecValTok{1}
\end{Highlighting}
\end{Shaded}

\begin{verbatim}
## [1] 0.2336781
\end{verbatim}

\begin{Shaded}
\begin{Highlighting}[]
\FunctionTok{check\_overdispersion}\NormalTok{(m3\_nb)}
\end{Highlighting}
\end{Shaded}

\begin{verbatim}
## # Overdispersion test
## 
##  dispersion ratio = 1.398
##           p-value = 0.088
\end{verbatim}

\begin{Shaded}
\begin{Highlighting}[]
\FunctionTok{check\_zeroinflation}\NormalTok{(m3\_nb)}
\end{Highlighting}
\end{Shaded}

\begin{verbatim}
## # Check for zero-inflation
## 
##    Observed zeros: 27
##   Predicted zeros: 30
##             Ratio: 1.12
\end{verbatim}

\begin{Shaded}
\begin{Highlighting}[]
\FunctionTok{check\_predictions}\NormalTok{(m3\_nb)}
\end{Highlighting}
\end{Shaded}

\includegraphics{hw1-lobstrs-eds241_files/figure-latex/unnamed-chunk-22-1.pdf}

\begin{Shaded}
\begin{Highlighting}[]
\FunctionTok{check\_model}\NormalTok{(m3\_nb)}
\end{Highlighting}
\end{Shaded}

\includegraphics{hw1-lobstrs-eds241_files/figure-latex/unnamed-chunk-23-1.pdf}

\begin{center}\rule{0.5\linewidth}{0.5pt}\end{center}

Step 6: Compare models

\textbf{a.} Use the \texttt{export\_summ()} function from the
\texttt{jtools} package to look at the three regression models you fit
side-by-side.

\textbf{c.} Write a short paragraph comparing the results. Is the
treatment effect \texttt{robust} or stable across the model
specifications.

\textbf{In the OLS model, there are about 5 more lobsters in MPA sites
than non-MPA sites, or a 23.6\% increase in lobsters in MPA sites. In
the Poisson and the negative binomial model, there is a 23.4\% increase
in lobsters in MPA sites. The treatment effect is stable across the
model specifications.}

\begin{Shaded}
\begin{Highlighting}[]
\FunctionTok{export\_summs}\NormalTok{(m1\_ols, m2\_pois, m3\_nb,}
             \AttributeTok{model.names =} \FunctionTok{c}\NormalTok{(}\StringTok{"OLS"}\NormalTok{,}\StringTok{"Poisson"}\NormalTok{, }\StringTok{"NB"}\NormalTok{),}
             \AttributeTok{statistics =} \StringTok{"none"}\NormalTok{)}
\end{Highlighting}
\end{Shaded}

 
  \providecommand{\huxb}[2]{\arrayrulecolor[RGB]{#1}\global\arrayrulewidth=#2pt}
  \providecommand{\huxvb}[2]{\color[RGB]{#1}\vrule width #2pt}
  \providecommand{\huxtpad}[1]{\rule{0pt}{#1}}
  \providecommand{\huxbpad}[1]{\rule[-#1]{0pt}{#1}}

\begin{table}[ht]
\begin{centerbox}
\begin{threeparttable}
 \setlength{\tabcolsep}{0pt}
\begin{tabular}{l l l l}


\hhline{>{\huxb{0, 0, 0}{0.8}}->{\huxb{0, 0, 0}{0.8}}->{\huxb{0, 0, 0}{0.8}}->{\huxb{0, 0, 0}{0.8}}-}
\arrayrulecolor{black}

\multicolumn{1}{!{\huxvb{0, 0, 0}{0}}c!{\huxvb{0, 0, 0}{0}}}{\huxtpad{6pt + 1em}\centering \hspace{6pt}  \hspace{6pt}\huxbpad{6pt}} &
\multicolumn{1}{c!{\huxvb{0, 0, 0}{0}}}{\huxtpad{6pt + 1em}\centering \hspace{6pt} OLS \hspace{6pt}\huxbpad{6pt}} &
\multicolumn{1}{c!{\huxvb{0, 0, 0}{0}}}{\huxtpad{6pt + 1em}\centering \hspace{6pt} Poisson \hspace{6pt}\huxbpad{6pt}} &
\multicolumn{1}{c!{\huxvb{0, 0, 0}{0}}}{\huxtpad{6pt + 1em}\centering \hspace{6pt} NB \hspace{6pt}\huxbpad{6pt}} \tabularnewline[-0.5pt]


\hhline{>{\huxb{255, 255, 255}{0.4}}->{\huxb{0, 0, 0}{0.4}}->{\huxb{0, 0, 0}{0.4}}->{\huxb{0, 0, 0}{0.4}}-}
\arrayrulecolor{black}

\multicolumn{1}{!{\huxvb{0, 0, 0}{0}}l!{\huxvb{0, 0, 0}{0}}}{\huxtpad{6pt + 1em}\raggedright \hspace{6pt} (Intercept) \hspace{6pt}\huxbpad{6pt}} &
\multicolumn{1}{r!{\huxvb{0, 0, 0}{0}}}{\huxtpad{6pt + 1em}\raggedleft \hspace{6pt} 22.73 *** \hspace{6pt}\huxbpad{6pt}} &
\multicolumn{1}{r!{\huxvb{0, 0, 0}{0}}}{\huxtpad{6pt + 1em}\raggedleft \hspace{6pt} 3.12 *** \hspace{6pt}\huxbpad{6pt}} &
\multicolumn{1}{r!{\huxvb{0, 0, 0}{0}}}{\huxtpad{6pt + 1em}\raggedleft \hspace{6pt} 3.12 *** \hspace{6pt}\huxbpad{6pt}} \tabularnewline[-0.5pt]


\hhline{}
\arrayrulecolor{black}

\multicolumn{1}{!{\huxvb{0, 0, 0}{0}}l!{\huxvb{0, 0, 0}{0}}}{\huxtpad{6pt + 1em}\raggedright \hspace{6pt}  \hspace{6pt}\huxbpad{6pt}} &
\multicolumn{1}{r!{\huxvb{0, 0, 0}{0}}}{\huxtpad{6pt + 1em}\raggedleft \hspace{6pt} (3.57)\hphantom{0}\hphantom{0}\hphantom{0} \hspace{6pt}\huxbpad{6pt}} &
\multicolumn{1}{r!{\huxvb{0, 0, 0}{0}}}{\huxtpad{6pt + 1em}\raggedleft \hspace{6pt} (0.02)\hphantom{0}\hphantom{0}\hphantom{0} \hspace{6pt}\huxbpad{6pt}} &
\multicolumn{1}{r!{\huxvb{0, 0, 0}{0}}}{\huxtpad{6pt + 1em}\raggedleft \hspace{6pt} (0.12)\hphantom{0}\hphantom{0}\hphantom{0} \hspace{6pt}\huxbpad{6pt}} \tabularnewline[-0.5pt]


\hhline{}
\arrayrulecolor{black}

\multicolumn{1}{!{\huxvb{0, 0, 0}{0}}l!{\huxvb{0, 0, 0}{0}}}{\huxtpad{6pt + 1em}\raggedright \hspace{6pt} treat \hspace{6pt}\huxbpad{6pt}} &
\multicolumn{1}{r!{\huxvb{0, 0, 0}{0}}}{\huxtpad{6pt + 1em}\raggedleft \hspace{6pt} 5.36\hphantom{0}\hphantom{0}\hphantom{0}\hphantom{0} \hspace{6pt}\huxbpad{6pt}} &
\multicolumn{1}{r!{\huxvb{0, 0, 0}{0}}}{\huxtpad{6pt + 1em}\raggedleft \hspace{6pt} 0.21 *** \hspace{6pt}\huxbpad{6pt}} &
\multicolumn{1}{r!{\huxvb{0, 0, 0}{0}}}{\huxtpad{6pt + 1em}\raggedleft \hspace{6pt} 0.21\hphantom{0}\hphantom{0}\hphantom{0}\hphantom{0} \hspace{6pt}\huxbpad{6pt}} \tabularnewline[-0.5pt]


\hhline{}
\arrayrulecolor{black}

\multicolumn{1}{!{\huxvb{0, 0, 0}{0}}l!{\huxvb{0, 0, 0}{0}}}{\huxtpad{6pt + 1em}\raggedright \hspace{6pt}  \hspace{6pt}\huxbpad{6pt}} &
\multicolumn{1}{r!{\huxvb{0, 0, 0}{0}}}{\huxtpad{6pt + 1em}\raggedleft \hspace{6pt} (5.20)\hphantom{0}\hphantom{0}\hphantom{0} \hspace{6pt}\huxbpad{6pt}} &
\multicolumn{1}{r!{\huxvb{0, 0, 0}{0}}}{\huxtpad{6pt + 1em}\raggedleft \hspace{6pt} (0.03)\hphantom{0}\hphantom{0}\hphantom{0} \hspace{6pt}\huxbpad{6pt}} &
\multicolumn{1}{r!{\huxvb{0, 0, 0}{0}}}{\huxtpad{6pt + 1em}\raggedleft \hspace{6pt} (0.17)\hphantom{0}\hphantom{0}\hphantom{0} \hspace{6pt}\huxbpad{6pt}} \tabularnewline[-0.5pt]


\hhline{>{\huxb{0, 0, 0}{0.8}}->{\huxb{0, 0, 0}{0.8}}->{\huxb{0, 0, 0}{0.8}}->{\huxb{0, 0, 0}{0.8}}-}
\arrayrulecolor{black}

\multicolumn{4}{!{\huxvb{0, 0, 0}{0}}l!{\huxvb{0, 0, 0}{0}}}{\huxtpad{6pt + 1em}\raggedright \hspace{6pt}  *** p $<$ 0.001;  ** p $<$ 0.01;  * p $<$ 0.05. \hspace{6pt}\huxbpad{6pt}} \tabularnewline[-0.5pt]


\hhline{}
\arrayrulecolor{black}
\end{tabular}
\end{threeparttable}\par\end{centerbox}

\end{table}
 

\begin{center}\rule{0.5\linewidth}{0.5pt}\end{center}

Step 7: Building intuition - fixed effects

\textbf{a.} Create new \texttt{df} with the \texttt{year} variable
converted to a factor

\textbf{b.} Run the following OLS model using \texttt{glm.nb()}

\begin{itemize}
\tightlist
\item
  Add fixed effects for \texttt{year} (i.e., dummy coefficients)
\item
  Include an interaction term between variables \texttt{treat} and
  \texttt{year}
\end{itemize}

\textbf{c.} Take a look at the regression output. Each coefficient
provides a comparison or the difference in means for a specific
sub-group in the data. Informally, describe the what the model has
estimated at a conceptual level (NOTE: you do not have to interpret
coefficients individually)

\begin{itemize}
\tightlist
\item
  \textbf{The model is estimating the effect that the treatment has on
  lobster counts for each year, from 2012 - 2018. For each year, it is
  estimating the mean count in both the treatment and the control
  groups.}
\end{itemize}

\textbf{d.} Explain why the main effect for treatment is negative? *Does
this result make sense?

\begin{itemize}
\tightlist
\item
  \textbf{The coefficient for treat represents the difference in lobster
  counts in MPA sites vs non-MPA sites in 2012. Since it is negative,
  the model estimates that there were less lobsters in MPA sites than
  non-MPA sites in 2012. Seeing that MPA sites were not designated in
  2012, it makes sense that we would not see an increase in lobster
  counts right away.}
\end{itemize}

\begin{Shaded}
\begin{Highlighting}[]
\NormalTok{ff\_counts }\OtherTok{\textless{}{-}}\NormalTok{ spiny\_counts }\SpecialCharTok{\%\textgreater{}\%} 
    \FunctionTok{mutate}\NormalTok{(}\AttributeTok{year=}\FunctionTok{as\_factor}\NormalTok{(year))}
    
\NormalTok{m5\_fixedeffs }\OtherTok{\textless{}{-}} \FunctionTok{glm.nb}\NormalTok{(}
\NormalTok{    count }\SpecialCharTok{\textasciitilde{}}
\NormalTok{        treat }\SpecialCharTok{+}
\NormalTok{        year }\SpecialCharTok{+}
\NormalTok{        treat}\SpecialCharTok{*}\NormalTok{year,}
    \AttributeTok{data =}\NormalTok{ ff\_counts)}

\FunctionTok{summ}\NormalTok{(m5\_fixedeffs, }\AttributeTok{model.fit =} \ConstantTok{FALSE}\NormalTok{)}
\end{Highlighting}
\end{Shaded}

\begin{table}[!h]
\centering
\begin{tabular}{lr}
\toprule
\cellcolor{gray!10}{Observations} & \cellcolor{gray!10}{252}\\
Dependent variable & count\\
\cellcolor{gray!10}{Type} & \cellcolor{gray!10}{Generalized linear model}\\
Family & Negative Binomial(0.8129)\\
\cellcolor{gray!10}{Link} & \cellcolor{gray!10}{log}\\
\bottomrule
\end{tabular}
\end{table}  \begin{table}[!h]
\centering
\begin{threeparttable}
\begin{tabular}{lrrrr}
\toprule
  & Est. & S.E. & z val. & p\\
\midrule
\cellcolor{gray!10}{(Intercept)} & \cellcolor{gray!10}{2.35} & \cellcolor{gray!10}{0.26} & \cellcolor{gray!10}{8.89} & \cellcolor{gray!10}{0.00}\\
treat & -1.72 & 0.42 & -4.12 & 0.00\\
\cellcolor{gray!10}{year2013} & \cellcolor{gray!10}{-0.35} & \cellcolor{gray!10}{0.38} & \cellcolor{gray!10}{-0.93} & \cellcolor{gray!10}{0.35}\\
year2014 & 0.08 & 0.37 & 0.21 & 0.84\\
\cellcolor{gray!10}{year2015} & \cellcolor{gray!10}{0.86} & \cellcolor{gray!10}{0.37} & \cellcolor{gray!10}{2.32} & \cellcolor{gray!10}{0.02}\\
\addlinespace
year2016 & 0.90 & 0.37 & 2.43 & 0.01\\
\cellcolor{gray!10}{year2017} & \cellcolor{gray!10}{1.56} & \cellcolor{gray!10}{0.37} & \cellcolor{gray!10}{4.25} & \cellcolor{gray!10}{0.00}\\
year2018 & 1.04 & 0.37 & 2.81 & 0.00\\
\cellcolor{gray!10}{treat:year2013} & \cellcolor{gray!10}{1.52} & \cellcolor{gray!10}{0.57} & \cellcolor{gray!10}{2.66} & \cellcolor{gray!10}{0.01}\\
treat:year2014 & 2.14 & 0.56 & 3.80 & 0.00\\
\addlinespace
\cellcolor{gray!10}{treat:year2015} & \cellcolor{gray!10}{2.12} & \cellcolor{gray!10}{0.56} & \cellcolor{gray!10}{3.79} & \cellcolor{gray!10}{0.00}\\
treat:year2016 & 1.40 & 0.56 & 2.50 & 0.01\\
\cellcolor{gray!10}{treat:year2017} & \cellcolor{gray!10}{1.55} & \cellcolor{gray!10}{0.56} & \cellcolor{gray!10}{2.77} & \cellcolor{gray!10}{0.01}\\
treat:year2018 & 2.62 & 0.56 & 4.69 & 0.00\\
\bottomrule
\end{tabular}
\begin{tablenotes}
\item Standard errors: MLE
\end{tablenotes}
\end{threeparttable}
\end{table}

\textbf{e.} Look at the model predictions: Use the
\texttt{interact\_plot()} function from package \texttt{interactions} to
plot mean predictions by year and treatment status.

\textbf{f.} Re-evaluate your responses (c) and (d) above.

\begin{itemize}
\tightlist
\item
  \textbf{The plot affirms my responses from above. The model is
  estimating the treatment effect on lobster counts in MPA sites
  vs.~non-MPA sites. The negative coefficients are telling us that in
  2012 and 2013, there were less lobsters in the newly designated MPA
  sites than non-MPA sites. Looking at the plot, we can see that this
  changes over time and eventually, there are more lobsters in MPA sites
  than non-MPA sites.}
\end{itemize}

\begin{Shaded}
\begin{Highlighting}[]
\CommentTok{\# Hint 1: Group counts by \textasciigrave{}year\textasciigrave{} and \textasciigrave{}mpa\textasciigrave{} and calculate the \textasciigrave{}mean\_count\textasciigrave{}}
\CommentTok{\# Hint 2: Convert variable \textasciigrave{}year\textasciigrave{} to a factor}

\FunctionTok{interact\_plot}\NormalTok{(m5\_fixedeffs, }\AttributeTok{pred =}\NormalTok{ year, }\AttributeTok{modx =}\NormalTok{ treat,}
              \AttributeTok{outcome.scale =} \StringTok{"response"}\NormalTok{)  }\CommentTok{\# }\AlertTok{NOTE}\CommentTok{: y{-}axis on log{-}scale}
\end{Highlighting}
\end{Shaded}

\includegraphics{hw1-lobstrs-eds241_files/figure-latex/unnamed-chunk-26-1.pdf}

\begin{Shaded}
\begin{Highlighting}[]
\CommentTok{\# HINT: Change \textasciigrave{}outcome.scale\textasciigrave{} to "response" to convert y{-}axis to counts}
\end{Highlighting}
\end{Shaded}

\textbf{g.} Using \texttt{ggplot()} create a plot in same style as the
previous \texttt{interaction\ plot}, but displaying the original scale
of the outcome variable (lobster counts). This type of plot is commonly
used to show how the treatment effect changes across discrete time
points (i.e., panel data).

The plot should have\ldots{} - \texttt{year} on the x-axis -
\texttt{counts} on the y-axis - \texttt{mpa} as the grouping variable

\begin{Shaded}
\begin{Highlighting}[]
\CommentTok{\# Hint 1: Group counts by \textasciigrave{}year\textasciigrave{} and \textasciigrave{}mpa\textasciigrave{} and calculate the \textasciigrave{}mean\_count\textasciigrave{}}
\CommentTok{\# Hint 2: Convert variable \textasciigrave{}year\textasciigrave{} to a factor}

\NormalTok{plot\_counts }\OtherTok{\textless{}{-}}\NormalTok{ spiny\_counts }\SpecialCharTok{|\textgreater{}}
    \FunctionTok{group\_by}\NormalTok{(year, mpa) }\SpecialCharTok{|\textgreater{}} 
    \FunctionTok{summarise}\NormalTok{(}\AttributeTok{mean\_count =} \FunctionTok{mean}\NormalTok{(count, }\AttributeTok{na.rm =} \ConstantTok{TRUE}\NormalTok{)) }\SpecialCharTok{|\textgreater{}} 
    \FunctionTok{mutate}\NormalTok{(}\AttributeTok{year =} \FunctionTok{as\_factor}\NormalTok{(year)) }\SpecialCharTok{|\textgreater{}}
    \FunctionTok{ungroup}\NormalTok{()}

\NormalTok{plot\_counts }\SpecialCharTok{|\textgreater{}}  
    \FunctionTok{ggplot}\NormalTok{(}\FunctionTok{aes}\NormalTok{(}\AttributeTok{x =}\NormalTok{ year, }\AttributeTok{y =}\NormalTok{ mean\_count, }\AttributeTok{group =}\NormalTok{ mpa, }\AttributeTok{color =}\NormalTok{ mpa, }\AttributeTok{linetype =}\NormalTok{ mpa)) }\SpecialCharTok{+}
    \FunctionTok{geom\_point}\NormalTok{(}\AttributeTok{size =} \DecValTok{4}\NormalTok{) }\SpecialCharTok{+}
    \FunctionTok{geom\_line}\NormalTok{(}\AttributeTok{size =} \DecValTok{1}\NormalTok{) }\SpecialCharTok{+}
    \FunctionTok{scale\_color\_manual}\NormalTok{(}\AttributeTok{values =} \FunctionTok{c}\NormalTok{(}\StringTok{"\#1D4677"}\NormalTok{, }\StringTok{"\#A9D1F6"}\NormalTok{ ),}
                       \AttributeTok{labels =} \FunctionTok{c}\NormalTok{(}\StringTok{"MPA"}\NormalTok{, }\StringTok{"Non{-}MPA"}\NormalTok{)) }\SpecialCharTok{+}
    \FunctionTok{scale\_linetype\_manual}\NormalTok{(}\AttributeTok{values =} \FunctionTok{c}\NormalTok{(}\StringTok{"solid"}\NormalTok{, }\StringTok{"longdash"}\NormalTok{),}
                          \AttributeTok{labels =} \FunctionTok{c}\NormalTok{(}\StringTok{"MPA"}\NormalTok{, }\StringTok{"Non{-}MPA"}\NormalTok{)) }\SpecialCharTok{+}
    \FunctionTok{theme\_minimal}\NormalTok{() }\SpecialCharTok{+}
    \FunctionTok{labs}\NormalTok{(}\AttributeTok{title =} \StringTok{"Mean lobster counts in MPA vs non{-}MPA designated reefs from 2012 {-} 2018"}\NormalTok{,}
         \AttributeTok{color =} \StringTok{"MPA designation"}\NormalTok{,}
         \AttributeTok{linetype =} \StringTok{"MPA designation"}\NormalTok{) }\SpecialCharTok{+}
    \FunctionTok{xlab}\NormalTok{(}\StringTok{"Year"}\NormalTok{) }\SpecialCharTok{+}
    \FunctionTok{ylab}\NormalTok{(}\StringTok{"Mean Lobster Count"}\NormalTok{)}
\end{Highlighting}
\end{Shaded}

\includegraphics{hw1-lobstrs-eds241_files/figure-latex/unnamed-chunk-27-1.pdf}

\begin{center}\rule{0.5\linewidth}{0.5pt}\end{center}

Step 8: Reconsider causal identification assumptions

\begin{enumerate}
\def\labelenumi{\alph{enumi}.}
\tightlist
\item
  Discuss whether you think \texttt{spillover\ effects} are likely in
  this research context (see Glossary of terms;
  \url{https://docs.google.com/document/d/1RIudsVcYhWGpqC-Uftk9UTz3PIq6stVyEpT44EPNgpE/edit?usp=sharing})
\end{enumerate}

\begin{itemize}
\tightlist
\item
  \textbf{I think it is possible that spillover effects occurred in this
  context! This seems like a debatable topic, but I think it is entirely
  possible that increased lobster population sizes in our MPA sites
  could spill over the invisible boundary of the MPAs and into our other
  reef sites. The sites are not that far apart, and lobsters are able to
  travel that far. However, looking at the plot above of mean lobster
  counts, the decrease in lobsters at Non-MPA sites from 2017-2018
  compared to the sharp increase in MPA sites during that same time
  indicate that an abundance of lobsters in MPA sites may not be
  affecting counts in non-MPA sites.}
\end{itemize}

\begin{enumerate}
\def\labelenumi{\alph{enumi}.}
\setcounter{enumi}{1}
\tightlist
\item
  Explain why spillover is an issue for the identification of causal
  effects
\end{enumerate}

\begin{itemize}
\tightlist
\item
  \textbf{In identifying causal effects, we really want our treatment
  and control groups to be independent of each other. If there is
  spillover and the treatment group is affecting the outcome of the
  control, then we will not have an unbiased baseline to compare our
  treatment outcomes to. This is an issue because we will not be able to
  measure the true causal effect.}
\end{itemize}

\begin{enumerate}
\def\labelenumi{\alph{enumi}.}
\setcounter{enumi}{2}
\tightlist
\item
  How does spillover relate to impact in this research setting?
\end{enumerate}

\begin{itemize}
\tightlist
\item
  \textbf{In this setting, spillover from MPA sites to Non-MPA sites has
  a positive impact. Though it may affect the results of our study, it
  is a good thing that there may be more lobsters in all sites because
  of the effects from the MPA sites.}
\end{itemize}

\begin{enumerate}
\def\labelenumi{\alph{enumi}.}
\setcounter{enumi}{3}
\item
  Discuss the following causal inference assumptions in the context of
  the MPA treatment effect estimator. Evaluate if each of the assumption
  are reasonable:

  \begin{enumerate}
  \def\labelenumii{\arabic{enumii})}
  \item
    SUTVA: Stable Unit Treatment Value assumption

    \begin{itemize}
    \tightlist
    \item
      \textbf{In our case, the SUTVA implies that MPA treatment effects
      are applied equally to all lobsters within MPAs, and that lobsters
      in the control groups do not recieve any affect from neighboring
      MPAs. This assumption would be violated if there is in fact a
      spillover effect occurring in our experiment, however without
      direct proof that spillover occurs, I think that the assumption is
      reasonable.}
    \end{itemize}
  \item
    Excludability assumption
  \end{enumerate}

  \begin{itemize}
  \tightlist
  \item
    \textbf{The excludability assumption implies that the MPA treatment
    is the sole causal effect on the outcome. In this case, it is very
    possible that there are outside environmental factors at play that
    effect lobster counts, like warming. However, considering that our
    sites are fairly close together, any environmental effects at play
    should affect all of our sites, and therefore not change any
    comparisons between our treatment and control sites. For this
    reason, I think that excludability is a fair assumption.}
  \end{itemize}
\end{enumerate}

\begin{center}\rule{0.5\linewidth}{0.5pt}\end{center}

\section{EXTRA CREDIT}\label{extra-credit}

\begin{quote}
Use the recent lobster abundance data with observations collected up
until 2024 (\texttt{lobster\_sbchannel\_24.csv}) to run an analysis
evaluating the effect of MPA status on lobster counts using the same
focal variables.
\end{quote}

\begin{enumerate}
\def\labelenumi{\alph{enumi}.}
\tightlist
\item
  Create a new script for the analysis on the updated data
\item
  Run at least 3 regression models \& assess model diagnostics
\item
  Compare and contrast results with the analysis from the 2012-2018 data
  sample (\textasciitilde{} 2 paragraphs)
\end{enumerate}

\begin{center}\rule{0.5\linewidth}{0.5pt}\end{center}

\includegraphics{figures/spiny1.png}

\end{document}
